\textbf{[BOJ 스택에 있는 지문이 최신 버전입니다. 지문은 BOJ 스택에서 확인해주세요]}


정환은 '피라미드'라는 게임판을 만들었다. 크기가 $N$인 피라미드는 $N$층으로 이루어져 있으며 $i$층은 $N-i+1$칸으로 나뉘어 있는 삼각형 모양의 게임판이다. 피라미드 $i$층의 $j$번째 칸을 $A_{i,j}$라 하고, 가장 꼭대기에 있는 칸 $A_{N,1}$을 피라미드의 탑, 피라미드 안에서 찾을 수 있는 크기가 작거나 같은 피라미드를 부분피라미드라고 정의했다.

정환은 크기가 $N$인 피라미드의 각 칸에 음이 아닌 정수를 적어두고 다음과 같은 2인용 게임을 기획했다.

--- 먼저 양의 정수 $K$를 정하고 게임을 시작한다. 두 플레이어는 서로 턴을 번갈아 가면서 게임을 진행하고, 자신의 턴에 행동을 할 수 없는 플레이어가 패배한다.

--- 각 턴에는 정확히 한 번, 피라미드에서 크기가 $K$인 부분피라미드와 임의의 양의 정수 $P$를 고르고 그 부분피라미드의 모든 칸에 $P$를 XOR하고 턴을 넘긴다. \textbf{단 부분피라미드의 탑에 적힌 수가 감소하는 행동에 한해서만 할 수 있다.}

게임은 완성했지만 게임에 대한 오랜 연구로 피라미드의 일부 칸이 노후화되어 쓰지 못하게 되었다. 아쉬운대로 노후화된 칸들은 사용하지 않기로 했다. 즉, 노후화된 칸을 탑으로 하는 부분피라미드를 고를 수 없고 노후화된 칸에는 $P$를 XOR하지 않는다.

$T$개의 테스트케이스에 대해 크기가 $N$인 피라미드와 $K$가 주어졌을 때 선공과 후공 중 누가 필승법을 가지고 있는 지 구해보자.
