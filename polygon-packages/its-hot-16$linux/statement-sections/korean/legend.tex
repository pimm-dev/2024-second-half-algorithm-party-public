한 여름의 날씨는 더워도 너무 덥다. 외출은 커녕 집에서도 에어컨을 끄지 못하지만, 그럼에도 민규는 바쁜 약속을 위해 약속 장소로 발걸음을 옮겨야 한다. 

밖은 $N$행 $M$열의 격자로 이루어져 있다. 격자의 각 칸은 집, 약속 장소, 건물, 벽, 길 중 하나로 이루어져 있다.

밖은 너무 더워 길 위에 있을 때는 지속적으로 불쾌함이 증가한다. 그러나 중간 중간 이동 경로에 있는 건물은 매우 시원하여, 건물을 지나가거나 건물 내에서 쉬는 동안에는 불쾌함이 감소한다.

민규의 불쾌함은 $0$으로 시작하며, $100$이 되면 견딜 수 없어 쓰러지고 만다.

민규는 매 시간 다음과 같은 순서로 행동한다. 

\begin{enumerate}
  \item 격자판 내에서 벽이 아닌 인접한 칸으로 이동하거나, 이동하지 않고 가만히 있는다.

  \item 현재 있는 칸에 따라서 불쾌함이 증가하거나 감소한다.
  \begin{itemize}
    \item 현재 있는 칸이 건물이라면 불쾌함이 $K$만큼 감소한다. 불쾌함은 $0$ 미만으로 내려가지 않는다.
    \item 현재 있는 칸이 건물이 아니라면 불쾌함이 $C$만큼 증가한다. 불쾌함이 $100$ 이상이 되면 쓰러지며 더 이상 행동하지 못한다.
  \end{itemize}

  \item 약속 장소에 도달하면 행동을 멈춘다.
\end{enumerate}

민규가 약속 장소에 쓰러지지 않고 도달할 수 있도록, 민규의 이동 경로를 계산하는 프로그램을 작성해주자.
