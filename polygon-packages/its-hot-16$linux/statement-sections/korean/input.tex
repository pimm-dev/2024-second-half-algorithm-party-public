첫 번째 줄에 격자의 크기를 나타내는 정수 $N$, $M$이 공백으로 구분되어 주어진다. ($2 \leq N, M \leq 100$)
두 번째 줄에 건물에서 감소하는 불쾌함, 길에서 증가하는 불쾌함을 나타내는 정수 $K$, $C$가 공백으로 구분되어 주어진다. ($1 \leq K, C \leq 100$)
세 번째 줄부터 $N+2$번째 줄까지 거리의 상태를 나타내는 길이 $M$의 문자열이 주어진다. $i+2$번째 줄의 $j$번째 문자는 $i$행 $j$열의 상태를 나타낸다. 만약 문자가 \t{S}라면 민규의 집 위치, \t{E}라면 약속 장소, \t{H}라면 건물, \t{#}이라면 벽, \t{.}이라면 길을 의미한다. \t{S}와 \t{E}는 하나만 주어진다.
