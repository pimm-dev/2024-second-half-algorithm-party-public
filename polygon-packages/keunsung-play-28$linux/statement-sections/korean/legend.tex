\it{근성은 나무에 관심이 많다.}

평소 나무를 깨끗이 관리해온 근성은 그 능력을 인정받아 북구청 청소행정과에 근무하게 되었다. 어느 날, 민규는 근성을 일하게 하기 위해 나무가 있는 위치에 쓰레기를 버리려고 한다. 나무는 수직선과 같은 일직선상에 있고 근성의 현재 위치는 원점이다. 아래의 두 쿼리를 수행할 때 근성의 총 이동 거리를 구하는 프로그램을 작성하시오.

--- $1$ $X$ : 민규가 좌표 $X$에 있는 나무에 쓰레기를 버린다.

--- $2$ : 근성이 현재 위치에서 시작하여 가장 가까운 쓰레기가 있는 나무의 좌표로 이동하여 쓰레기를 수거하며 모든 쓰레기를 수거할 때까지 이 행동을 반복한다. 단, 근성의 현재 위치에서 가장 가까운 좌표가 두 개 이상일 시 가장 작은 좌표로 이동한다.
