대학 졸업을 앞둔 전남대의 고학번 근성도 어느덧 훈련소에 갈 시기가 되었다.

훈련소에 입소한 첫날밤.  대학생활을 되돌아보며 너무나 많은 고마운 사람이 있다는 것을 깨닫게된 근성은 고마운 사람들의 이름이 담긴 편지를 적기로 마음 먹었다.

근성이 입소할 당시에는 훈련병이 스마트폰을 사용해 연락을 할 수 있었지만, 그 방식으로는 고마움이 담기지 않는다 생각한 근성은 열심히 손편지를 적기 시작하였다. 

감성 한 스푼, 고마움 두 스푼이 담긴 편지를 다 적은 후 근성은 편지의 발송을 요청하였지만 편지담당자의 실수로 편지의 일부 몇 장이 누락되고 말았다!

편지에 고마움이 대상에게 전달 될 수 있으려면 다음 조건을 만족해야한다.

\begin{itemize}
\item 편지지 하나에 이름이 온전하게 있다면 **등장**하는 것이다.
\begin{itemize}
    \item 1번 편지지 : `김근성` 이 내용에 들어있다.
    \item 1번 편지지에 `김근성`이 있기에 김근성은 편지에 등장한다.
\end{itemize}
\item 여러장의 편지지에 이름이 걸쳐 있어도 연속되게 있다면 **등장**하는 것이다.
\begin{itemize}
    \item 1번 편지지 : `김` 으로 편지내용이 끝난다.
    \item 2번 편지지 : `근성` 으로 편지내용이 시작한다.
    \item 1번 편지지와 2번 편지지 내용이 합쳐지면 `김근성` 이기에 김근성은 편지에 등장한다.
\end{itemize}
\item 만약 편지지가 누락되어서 이름이 연속되게 된다면 **등장**하는 것이다.
\begin{itemize}    
    \item 1번 편지지 : `김` 으로 편지내용이 끝난다.
    \item 2번 편지지 : 상관없는 내용이 적혀있다.
    \item 3번 편지지 : `근성` 으로 편지내용이 시작한다.
    \item 이때 2번 편지지가 누락되고 1번 편지지와 3번 편지지의 내용이 합쳐지만 `김근성`이기에 김근성은 편지에 등장한다.
\end{itemize}
\item N개의 편지지중 어떤 M개의 편지지가 누락되더라도 대상의 이름이 반드시 **등장**한다면 해당 대상에게는 고마움이 전달된다.
\end{itemize}

편지의 일부 몇 장이 누락되더라도 근성의 고마움이 대상에게 전달 될 수 있을지 구해보자
