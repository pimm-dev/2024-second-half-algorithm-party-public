게임개발동아리 PIMM에는 졸업생들이 전공책을 동아리방에 두고 가는 문화가 있다. PIMM에서는 신입 회원들이 전공책을 사용할 수 있도록 도서 대여 시스템을 운영했다. 하지만 민규가 회장직을 맡았던 작년에는 동아리방에 너무 많은 책이 쌓여 도서 대여 시스템을 관리하기 어렵게 되었다. 이 소식을 들은 종현은 필요한 책들을 바로바로 찾을 수 있도록 하는 도서 검색 프로그램 제작을 착수했다.

종현이 만든 도서 검색 프로그램 DB에는 문자열 \t{<book>}이 저장된다. \t{<book>}은 \href{https://ko.wikipedia.org/wiki/배커스-나우르_표기법}{배커스-나우르 표기법(BNF)}에 따라 다음과 같이 표기될 수 있다.

\begin{lstlisting}
<book> ::= <name> <author> | <name> <author> <tags>
<tags> ::= <tag> | <tag> <tags>

<name> ::= <str>
<author> ::= <str>
<tag> ::= <str>
\end{lstlisting}

\begin{itemize}
  \item \t{<str>}은 알파벳 대소문자, 숫자, '-', '_' 로만 이루어진 문자열이다.
  \item \t{<name>}은 도서의 이름을 의미한다.
  \item \t{<author>}은 도서의 저자를 의미한다. 
  \item \t{<tag>}는 도서가 가진 특징을 의미한다.
  \item \t{<name>}, \t{<author>}, \t{<tag>}의 최대 길이는 각각 20보다 작거나 같다.
\end{itemize}

만약 "1984 George_Orwell novel SF"라는 문자열 \t{<book>}이 DB 안에 있다면, 이것은 
이름이 "1984"이고 저자가 "George_Orwell"이며, "novel"과 "SF"라는 두 가지 특징을 지닌 도서가 도서 검색 프로그램 안에 있음을 의미한다.

DB 안에 \t{<name>} 과 \t{<author>} 가 모두 같은 책이 둘 이상 존재할 수 없다. 단, \t{<name>} 이나 \t{<author>} 둘 중 하나가 일치하는 책이 DB 안에 여러 권 있을 수 있음에 유의하라.


도서 검색 프로그램에 주어지는 검색 요청 \t{<search>} 또한 문자열로 주어진다.

\begin{lstlisting}
<method> ::= <AND> | <OR> | <NOT> | <nameCorrect> | <nameInclude> | <authorCorrect> | <tagCorrect>
<methods> ::= <method> | <method> ", " <methods>

<AND> ::= "A(" <methods> ")"
<OR> ::= "O(" <methods> ")"
<NOT> ::= "N(" <method> ")"
<nameCorrect> ::= "n:" <str>
<nameInclude> ::= "ni:" <str>
<authorCorrect> ::= "a:" <str>
<tagCorrect> ::= "t:" <str>

<search> ::= <methods>
\end{lstlisting}

\begin{itemize}
  \item \t{<nameCorrect>}
    \item \t{<str>}과 주어진 \t{<book>}의 \t{<name>}이 일치하면 참을, 그렇지 않다면 거짓을 반환한다.

  \item \t{<nameInclude>}
    \item \t{<str>}이 주어진 \t{<book>}의 문자열 \t{<name>} 안에 포함되어 있다면 참을, 그렇지 않다면 거짓을 반환한다.

  \item \t{<authorCorrect>}
    \item \t{<str>}과 주어진 \t{<book>}의 \t{<author>}과 일치하면 참을, 그렇지 않다면 거짓을 반환한다.

  \item \t{<tagCorrect>}
    \item \t{<str>}과 주어진 \t{<book>}의 \t{<tags>} 안에 일치하는 \t{<tag>}가 하나라도 있다면 참을, 그렇지 않다면 거짓을 반환한다.

  \item \t{<AND>}
    \item 주어진 \t{<methods>} 안의 모든 \t{<method>}가 참을 반환한다면 참을, 그렇지 않다면 거짓을 반환한다.

  \item \t{<OR>}
    \item 주어진 \t{<methods>} 안의 \t{<method>} 중 하나라도 참을 반환한다면 참을, 그렇지 않다면 거짓을 반환한다.

  \item \t{<NOT>}
    \item \t{<method>}가 참을 반환한다면 거짓을, 그렇지 않다면 참을 반환한다.

  \item \t{<search>}
    \item DB 안의 모든 \t{<book>}에 대해, 주어진 \t{<methods>} 안의 모든 \t{<method>}가 참을 반환하는 \t{<book>}의 개수를 반환한다.

  \item \t{<search>} 내에 들어가는 \t{<str>}의 최대 길이는 20보다 작거나 같다.
\end{itemize}

여러 개의 책 \t{<book>}과, 여러 개의 문자열 \t{<search>}가 주어질 때, 각각의 \t{<search>}에 대해 반환되는 값을 한 줄에 하나씩 출력하라.
